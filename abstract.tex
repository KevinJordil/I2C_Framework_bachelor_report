% Francais
Le projet vise le développement d'un \textit{framework} pour écosystème décentralisé sur I2C.
Dans ce contexte, le projet vise principalement à résoudre certains problèmes rencontrés au sein du club de robotique de la Haute Ecole d'Ingénierie et de Gestion du Canton de Vaud (HEIG-VD) lors du concours international de robotique amateur, Eurobot.
Il a également pour objectif de résoudre des problèmes dans des projets industriels menés au sein de l'institut Reconfigurable \& embedded Digital Systems (REDS) et de l'Institut d'Automatisation industrielle (IAI).
L'objectif est d'améliorer l'efficacité et la réutilisabilité des composants tout en réduisant les coûts de développement et en optimisant les performances.

Le projet offre également une opportunité de relever les défis techniques et logistiques liés à la gestion des dispositifs, en combinant des aspects logiciels et matériels essentiels.
Le processus comprend une phase d'analyse, suivi d'une implémentation et enfin des tests, permettant ainsi la mise en place du \textit{framework} pour écosystème décentralisé sur I2C.

Au terme de ce développement, il est désormais possible de détecter les périphériques compatibles avec l'écosystème sur le bus I2C en affichant toutes les données correspondantes
De plus, il est maintenant réalisable de mettre à jour le micrologiciel des composants via le bus I2C.

\asterism

% English
The project aims at developing a framework for a decentralized ecosystem on I2C.
In this context, the main objective of the project is to address specific issues encountered within the Robotics Club of the Haute Ecole d’Ingénierie et de Gestion du Canton de Vaud (HEIG-VD) during the international amateur robotics competition, Eurobot.
Additionally, it aims to solve problems in industrial projects carried out within the Reconfigurable \& embedded Digital Systems (REDS) Institute and the Institut d'Automatisation industrielle (IAI).
The goal is to improve component efficiency and reusability while reducing development costs and optimizing performance.

It also offers an opportunity to tackle technical and logistical challenges related to device management, combining essential software and hardware aspects.
The process involves an analysis phase, followed by implementation and testing, ultimately leading to the establishment of the framework for the decentralized ecosystem on I2C.

After completing this development, it is now possible to detect devices compatible with the ecosystem on the I2C bus by displaying all corresponding data.
Furthermore, updating firmware for components through the I2C bus is now achievable.