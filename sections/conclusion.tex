Dans ce chapitre, l'objectif est de tirer une conclusion globale sur le projet, en examinant les résultats obtenus d'une part, et en partageant des réflexions personnelles d'autre part.

\section{Résultats}

Les résultats de l'évaluation, basée sur le cahier des charges initial, démontrent que toutes les tâches prévues ont été réalisées avec succès.
En outre, l'un des points supplémentaires prévus, à savoir la création d'un démonstrateur sous forme de circuit imprimé, a également été réalisé.
Ces accomplissements permettent au projet d'atteindre pleinement son objectif principal en fournissant un \textbf{\gls{framework} pour écosystème décentralisé sur \gls{i2c}}, en conformité avec les attentes établies.

Le travail effectué a permis de créer un code bien structuré et fonctionnel, qui peut être aisément utilisé dans d'autres projets.
La documentation technique qui l'accompagne fournit une explication claire et détaillée sur l'utilisation de l'écosystème développé, ce qui facilite sa réutilisation et son extension.
Ces éléments combinés garantissent que l'écosystème est désormais prêt à être exploité dans divers contextes, offrant ainsi une solution pratique et adaptée aux besoins des utilisateurs.

\section{Réflexions personnelles}

Ce projet de travail de bachelor s'est avéré être une expérience enrichissante qui m'a permis d'approfondir mes connaissances en développement de systèmes embarqués, notamment en explorant le protocole \gls{i2c} et en découvrant le domaine de l'électronique, qui n'est pas mis en avant dans la formation habituelle.
Cette opportunité de travailler avec le matériel a été particulièrement captivante, ce qui a motivé la création de ce projet.

L'utilisation de \LaTeX pour la rédaction de ce rapport a également été une nouveauté pour moi, et cela a nécessité des efforts pour parvenir à obtenir un environnement fonctionnel.
De même, c'était la première fois que je réalisais un projet d'une telle envergure de manière autonome, bénéficiant des ressources et du soutien de l'école.
Cette expérience a été extrêmement valorisante.

Au final, j'ai acquis une somme considérable de connaissances qui m'intéressent tant d'un point de vue personnel que professionnel.
Je suis satisfait du résultat obtenu, bien que le désir d'amélioration continue reste présent, comme c'est souvent le cas dans le développement de projets.


\vfil

\hspace{7cm}\begin{minipage}{5cm}
    \printsignature
\end{minipage}\par
\hspace{8cm}\makeatletter\@author\makeatother
\clearpage
%\hspace{8cm}\makeatletter\@author\makeatother\par
%\hspace{8cm}\begin{minipage}{5cm}
%    \printsignature
%\end{minipage}
%\clearpage