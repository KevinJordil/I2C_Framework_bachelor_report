Dans ce chapitre, l'objectif est d'examiner les aspects qui pourraient être améliorés dans le projet.
Il est courant de constater qu'il y a toujours des possibilités d'amélioration pour fournir un service encore plus satisfaisant.
Cependant, il est important de garder à l'esprit que le temps est limité et qu'un cahier des charges a été défini.
Malgré cela, il est tout de même intéressant de mettre en avant les points qui pourraient bénéficier d'améliorations futures.

\section{CI/CD}

L'intégration continue (CI) et le déploiement continu (CD) font partie des tâches potentielles du cahier des charges, si le temps le permet.
Ces pratiques pourraient être mises en place pour améliorer la rapidité du processus de développement.

Dans le cadre du club de robotique, l'utilisation de CI/CD pourrait être bénéfique.
Les informaticiens pourraient mettre à jour facilement le micrologiciel de certains composants en mettant à jour le code sur \textit{Github}.
Une fois que le fichier binaire du micrologiciel est généré, un signal pourrait être envoyé au maître I2C pour l'informer de la disponibilité de la mise à jour.
Cependant, certaines contraintes doivent être prises en compte.

Premièrement, le maître I2C doit avoir une connexion internet pour récupérer le fichier binaire généré par \textit{Github}.
Si l'accès à internet n'est pas disponible dans l'écosystème embarqué, le système CI/CD perdrait son utilité, et \textit{Github} serait simplement utilisé pour générer le fichier binaire.
Cette contrainte est importante à considérer dans le cadre d'un projet visant la réutilisation.

Deuxièmement, il est nécessaire de pouvoir identifier les périphériques à mettre à jour.
Lorsque le code d'un certain composant ou type de composant est mis à jour, il est logique de mettre à jour uniquement les périphériques concernés.
Cela implique la mise en place d'un CI/CD capable de différencier les périphériques en fonction des informations du dépôt \textit{Github}.
Bien que cela soit possible, la mise en place peut être complexe si des groupes de composants doivent être différenciés en cours de projet. Il serait probablement possible de spécifier à quel groupe, type ou adresse de périphérique il est nécessaire de déployer le nouveau micrologiciel.
Ainsi, lors de la récupération du fichier par le maître, celui-ci saurait comment déployer le micrologiciel.

En conclusion, l'utilisation de CI/CD peut être très utile dans certains contextes, mais elle n'est pas applicable dans d'autres en raison de la nécessité d'une connexion internet.
Il est donc essentiel d'évaluer attentivement les contraintes et besoins spécifiques avant de mettre en place cette amélioration.

\section{Interruption I2C}

Actuellement, le micrologiciel est configuré pour surveiller en permanence toute communication I2C qui lui est adressée, fonctionnant ainsi dans une boucle continue sans entrer en veille.
Bien que cela fonctionne parfaitement, il serait préférable d'utiliser une interruption lorsque l'adresse I2C correspond à celle du périphérique.
Cette approche permettrait de mettre en veille le périphérique et de le réveiller uniquement lorsqu'une communication le concerne.

Le \textit{\gls{framework}} Mbed ne propose pas directement la possibilité d'activer et d'utiliser les interruptions.
Cependant, il donne accès aux registres du microcontrôleur, permettant ainsi d'activer manuellement les interruptions, y compris celles de l'I2C.
Au cours de ce projet, plusieurs tentatives d'utilisation de cette interruption ont été effectuées.
Le déclenchement s'effectuait correctement les deux premières fois, mais ensuite le bus I2C se bloquait.
La raison de ce blocage n'a pas été identifiée.
Il ne s'agissait probablement pas d'un problème d'acquittement, car dans ce cas, le deuxième déclenchement ne se produirait pas.
Une analyse plus approfondie serait nécessaire pour comprendre le problème.

Lorsqu'une interruption I2C est mise en place, il est important de ne pas effectuer le traitement directement à l'intérieur de l'interruption.
Cela est une mauvaise pratique, car d'autres interruptions pourraient être bloquées.
Une approche plus appropriée consisterait à démarrer un thread lors de l'interruption, qui se chargerait de traiter toutes les opérations I2C nécessaires.
Cette approche permettrait de quitter rapidement l'interruption I2C tout en effectuant des traitements plus complexes en parallèle.
Mbed offre une gestion simplifiée des threads, ce qui rend cette mise en œuvre réalisable.

De plus, le microcontrôleur choisi possède une fonctionnalité intéressante.
Sur le bus I2C 1, il est possible de réveiller le microcontrôleur. Si le microcontrôleur est en veille et qu'une connexion I2C démarre, il peut sortir de veille automatiquement.
Cette possibilité permet de minimiser la consommation d'énergie tout en restant actif au moment opportun, ce qui est particulièrement intéressant pour un système embarqué.

\section{Option de redémarrage}

Dans les outils dédiés au maître I2C, un utilitaire existe pour mettre à jour le micrologiciel.
Une des options de ce script permet de spécifier l'état actuel du microcontrôleur, déterminant s'il se trouve dans le chargeur de démarrage ou dans l'application.
En cas de spécification incorrecte, la mise à jour ne fonctionnera pas correctement lorsque le microcontrôleur est en mode application, ou cela entraînera simplement une légère perte de temps s'il est mal spécifié pour le chargeur de démarrage.

Pour améliorer cela, il serait envisageable de réaliser une lecture sur l'appareil pour déterminer son état actuel.
Un nombre magique serait alors défini pour chaque état, permettant de redémarrer le microcontrôleur de manière appropriée pour être certain d'accéder au chargeur de démarrage et effectuer la mise à jour si nécessaire.
De cette manière, l'utilisateur n'aurait plus à se demander dans quel état se trouve le microcontrôleur actuellement.
Bien que cette amélioration ne soit pas indispensable, elle améliorerait l'expérience utilisateur.

\section{Bouton de mise à jour}

Une amélioration potentielle consisterait à remplacer le bouton actuel par un bouton qui serait connecté à une broche du microcontrôleur. Ainsi, lors du démarrage du microcontrôleur, le chargeur de démarrage pourrait vérifier si le bouton est enfoncé ou non.
Dans le cas où le bouton est pressé, le chargeur de démarrage pourrait attendre une mise à jour du micrologiciel.
Cela permettrait de mettre à jour le microcontrôleur même si un utilisateur a téléversé un code qui empêche de revenir au mode de mise à jour.
En conséquence, le débogueur ne serait utilisé qu'une seule fois dans la vie du périphérique.

Actuellement, cette amélioration n'est pas réalisable car le bouton existant sur le microcontrôleur est connecté à la broche de réinitialisation.
Lorsqu'il est enfoncé, cela redémarre le microcontrôleur de la même manière que lorsqu'il est allumé normalement.
Ce bouton est pratique lorsque l'on souhaite redémarrer rapidement un module, mais il ne permet pas de réaliser la fonctionnalité envisagée pour la mise à jour du micrologiciel.

\section{Connecteur non compatible}

Le circuit imprimé développé est équipé d'un connecteur doté de renforts sur les côtés.
Lors du choix de cette version du connecteur, l'idée était d'améliorer la solidité du connecteur.
Cependant, cette version du connecteur n'est plus entièrement compatible.
En effet, une petite partie doit être cassée pour pouvoir effectuer la connexion.
Cela n'affecte pas le fonctionnement, mais cela peut être déroutant pour l'utilisateur.
Deux solutions sont envisageables : adapter le connecteur sur le circuit imprimé ou adapter le connecteur au bout du câble.
Il serait plus approprié d'opter pour l'adaptation du connecteur sur le circuit imprimé afin de choisir la version sans renforts, qui est compatible.

