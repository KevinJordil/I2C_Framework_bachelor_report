Ce rapport a été réalisé dans le cadre d'un travail de bachelor à l'\gls{heig} en 2023.
Les objectifs définis dans le cahier des charges sont les suivants :

\begin{itemize}
    \item Un écosystème \gls{i2c} doit être développé pour les microcontrôleurs.
    \item Un \textit{\gls{framework}} de développement doit être choisi.
    \item Un microcontrôleur doit être sélectionné.
    \item Les éléments du bus doivent s'auto-adresser et être identifiés par un identifiant unique.
    \item Un chargeur de démarrage doit être mis en place pour permettre la mise à jour du micrologiciel embarqué.
    \item Une interface POSIX doit être développée pour interagir avec les éléments du bus.
    \item Un démonstrateur du système couvrant les points susmentionnés doit être réalisé.
    \item Le code doit être livré sur GitHub, accompagné d'une documentation adéquate.
\end{itemize}

Il existe également des objectifs qui peuvent être réalisés si le temps le permet :

\begin{itemize}
    \item Une carte électronique dédiée avec un capteur et/ou un afficheur doit être réalisée.
    \item La mise en place d'un CI/CD (Continuous Integration/Continuous Deployment).
\end{itemize}

