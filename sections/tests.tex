Dans ce chapitre, l'objectif est de démontrer que le projet fonctionne sur le démonstrateur réalisé.
Pour atteindre ce but, une multitude de scénarios de tests ont été mis en place afin de vérifier que toutes les fonctionnalités réagissent correctement.
Ces tests permettront de valider le bon fonctionnement de l'écosystème et de s'assurer qu'il répond aux spécifications établies dans le cahier des charges.

\section{Scénario de tests}

Lorsqu'un test est effectué sur un microcontrôleur, il fait référence au microcontrôleur choisi dans le cadre du projet sur le démonstrateur réalisé.

\subsection{Génération du micrologiciel}

\begin{table}[H]
    \begin{center}
        \caption{Tests : Génération du micrologiciel\label{tab:generationmicrologiciel}}
        \begin{tabularx}{\textwidth}{l|X|c}
            Description                & Résultat attendu                                                                               & Status    \\ \hline
            Compiler un micrologiciel. & L'entête inclut le nombre magique.                                                           & \checkmark \\
            Compiler un micrologiciel. & L'entête inclut la taille du micrologiciel, générée automatiquement.                   & \checkmark \\
            Compiler un micrologiciel. & L'entête inclut le \gls{crc} du micrologiciel, généré automatiquement.                & \checkmark \\
            Compiler un micrologiciel. & L'entête inclut le \textit{hash} de version du micrologiciel, généré automatiquement. & \checkmark
        \end{tabularx}
    \end{center}
\end{table}

\subsection{Chargeur de démarrage}

\begin{table}[H]
    \begin{center}
        \caption{Tests : Chargeur de démarrage\label{tab:chargeurdemarrage}}
        \begin{tabularx}{\textwidth}{X|X|c}
            Description                & Résultat attendu                                                                               & Status    \\ \hline
            Lorsque le microcontrôleur est équipé uniquement d'un chargeur de démarrage. & Le microcontrôleur reste en attente dans le chargeur de démarrage pour recevoir un micrologiciel. & \checkmark
        \end{tabularx}
    \end{center}
\end{table}

\subsection{Déploiement via I2C}

\begin{table}[H]
    \begin{center}
        \caption{Tests : Déploiement via \gls{i2c}\label{tab:deploiementi2c}}
        \begin{tabularx}{\textwidth}{X|X|c}
            Description                & Résultat attendu                                                                               & Status    \\ \hline
            Déployer un micrologiciel lorsque le microcontrôleur est dans l'application. & Le micrologiciel est téléversé. À la fin de l'envoi, le microcontrôleur démarre sur le micrologiciel. & \checkmark \\
            Déployer un micrologiciel lorsque le microcontrôleur est dans le chargeur de démarrage. & Le micrologiciel est téléversé. À la fin de l'envoi, le microcontrôleur démarre sur le micrologiciel. & \checkmark \\
            Déployer un micrologiciel sur le microcontrôleur. & Les métadonnées restent intactes. & \checkmark \\
            Arrêter le déploiement d'un micrologiciel. & Le microcontrôleur ne démarre pas sur un micrologiciel non valide. & \checkmark \\
            Déployer un micrologiciel sur un microcontrôleur grâce à son adresse. & Le périphérique ayant cette adresse est mis à jour. & \checkmark \\
            Déployer un micrologiciel sur un microcontrôleur grâce à son groupe. & Le périphérique ayant ce groupe est mis à jour. & \checkmark \\
            Déployer un micrologiciel sur des microcontrôleurs grâce à leur groupe. & Les périphériques ayant ce groupe sont mis à jour. & \checkmark \\
            Déployer un micrologiciel sur un microcontrôleur grâce à son type. & Le périphérique ayant ce type est mis à jour. & \checkmark \\
            Déployer un micrologiciel sur des microcontrôleurs grâce à son type. & Les périphériques ayant ce type sont mis à jour. & \checkmark \\
        \end{tabularx}
    \end{center}
\end{table}

\subsection{Déploiement via le débogueur}

Pour réaliser ces tests, il est nécessaire de connecter la sonde avec le connecteur Tag-Connect, ainsi que de fournir une alimentation.

\begin{table}[H]
    \begin{center}
        \caption{Tests : Déploiement via le débogueur\label{tab:deploiementdebogueur}}
        \begin{tabularx}{\textwidth}{X|X|c}
            Description                & Résultat attendu                                                                               & Status    \\ \hline
            Déployer le chargeur de démarrage uniquement. & Le chargeur de démarrage est téléversé et démarre, en attente d'un micrologiciel. & \checkmark \\
            Déployer le chargeur de démarrage avec une application. & Le chargeur de démarrage est téléversé avec une application et démarre, exécutant le micrologiciel. & \checkmark
        \end{tabularx}
    \end{center}
\end{table}

\subsection{Detection de périphériques}

\begin{table}[H]
    \begin{center}
        \caption{Tests : Detection de périphériques\label{tab:detectionpériphériques}}
        \begin{tabularx}{\textwidth}{X|X|c}
            Description                & Résultat attendu                                                                               & Status    \\ \hline
            Scanner le bus \gls{i2c} avec l'utilitaire linux \textit{i2cdetect} & Les périphériques de l'écosystème sont détectés normalement. & \checkmark \\
            Scanner le bus \gls{i2c} avec l'utilitaire réalisé dans ce projet. & Uniquement les périphériques compatibles avec l'écosystème sont détectés. & \checkmark \\
            Scanner le bus \gls{i2c} avec l'utilitaire réalisé dans ce projet. & L'identifiant unique de chaque périphérique compatible est affiché. & \checkmark \\
            Scanner le bus \gls{i2c} avec l'utilitaire réalisé dans ce projet. & L'adresse de chaque périphérique compatible est affichée. & \checkmark \\
            Scanner le bus \gls{i2c} avec l'utilitaire réalisé dans ce projet. & Le \textit{hash} du firmware de chaque périphérique compatible est affiché. & \checkmark \\
            Scanner le bus \gls{i2c} avec l'utilitaire réalisé dans ce projet. & Le groupe de chaque périphérique compatible est affiché. & \checkmark \\
            Scanner le bus \gls{i2c} avec l'utilitaire réalisé dans ce projet. & Le type de chaque périphérique compatible est affiché. & \checkmark \\
            Scanner le bus \gls{i2c} avec l'utilitaire réalisé dans ce projet. & Le nom de chaque périphérique compatible est affiché & \checkmark
        \end{tabularx}
    \end{center}
\end{table}

\subsection{Auto-adressage}

\begin{table}[H]
    \begin{center}
        \caption{Tests : Auto-adressage\label{tab:autoadressage}}
        \begin{tabularx}{\textwidth}{X|X|c}
            Description                & Résultat attendu                                                                               & Status    \\ \hline
            Déployer le micrologiciel sur un périphérique. & Le périphérique possède une adresse \gls{i2c}. & \checkmark \\
            Déployer le même micrologiciel sur deux périphériques. & Chaque périphérique possède une adresse \gls{i2c} différente. & \checkmark
        \end{tabularx}
    \end{center}
\end{table}

\subsection{Gestion des erreurs}

\begin{table}[H]
    \begin{center}
        \caption{Tests : Gestion des erreurs\label{tab:gestionerreurs}}
        \begin{tabularx}{\textwidth}{X|X|c}
            Description                & Résultat attendu                                                                               & Status    \\ \hline
            Mettre le signal SCL à l'état bas pendant plus de cinq secondes. & Les périphériques compatibles avec l'écosystème redémarrent. & \checkmark
        \end{tabularx}
    \end{center}
\end{table}

