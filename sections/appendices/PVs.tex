Dans un projet comme celui-ci, des séances sont régulièrement organisées avec le superviseur, Monsieur Yves Chevallier.
Pour conserver une trace des éléments importants discutés pendant le développement du projet, des procès-verbaux des séances sont réalisés.
L'objectif est de noter les points clés plutôt que chaque phrase ou question posée pendant les séances.

Pendant les semaines de travail à plein temps, aucune séance hebdomadaire n'était initialement prévue. Les séances étaient organisées au besoin uniquement, ce qui impliquait parfois des discussions rapides dans les couloirs ou dans le bureau, sans procès-verbaux.

Pendant les dernières semaines du travail de bachelor, Monsieur Yves Chevallier était en vacances, ce qui a entraîné des échanges de courriels au lieu de séances en personne.

\section{Vendredi 24 février 2023}

\subsection{Temps de travail}

Un travail de bachelor doit comporter 450 heures de travail, incluant une marge de 20\% pour les imprévus.
En raison de l'implication au sein du club de robotique, le travail de bachelor sera décalé de deux semaines et se terminera plus tard.
Le temps de travail pendant le semestre s'élèvera à 8.125 heures par semaine.
Il est également nécessaire de prévoir 50 heures pour la rédaction du rapport, ce qui inclut le rapport intermédiaire.

\subsection{Projet}

Il est essentiel de se questionner sur le projet et de prendre du recul pour proposer des solutions adaptées à chaque situation.
En déterminant le pourquoi et le comment, le cadre du projet pourra être fixé.

\subsection{Rendu}

D'ici la fin de la semaine cinq, un cahier des charges et une planification du projet doivent être rendus.
Cela permettra de générer le cahier des charges au format liste sur Gaps, ce qui sera réalisé lors d'une prochaine séance.

\subsection{Besoin du client}

Les besoins du client sont les suivants :
\begin{itemize}
    \item Assurer la pérennité du club de robotique en documentant le projet et en permettant sa réutilisation.
    \item Apporter un aspect technologique au sein du club de robotique, c'est-à-dire le projet en lui-même.
    \item Assurer un bon déroulement au sein du club de robotique.
    \item Apporter une utilité pour les projets au \gls{reds} et à l'\gls{iai}, en ayant un projet au sens large.
\end{itemize}

\subsection{Technologie}

Il est intéressant d'explorer en profondeur la piste des Nucleo STM32 pour le projet.

\section{Vendredi 3 mars 2023}

\subsection{Bus de communication}

\subsubsection{I2C}

Pour avoir un état haut lorsque le bus I2C est libre, une résistance de type \textit{pull-up} est utilisée.
Si l'I2C est choisi, il sera nécessaire de mettre en place une gestion des erreurs.
Pour la résistance de type \textit{pull-up}, une solution possible est d'utiliser un microcontrôleur pour la génération.

\subsubsection{CAN}

Le bus CAN utilise deux transistors pour faire passer le signal, contrairement à l'I2C qui utilise une résistance de type \textit{pull-up}.

La plupart des microcontrôleurs ne possèdent pas de transceiver CAN intégré, ce qui nécessitera l'ajout d'un composant supplémentaire sur le circuit imprimé.
Il existe peut-être quelques microcontrôleurs avec un transceiver intégré.
Il faudra mettre une résistance de 120 ohms à chaque bout de la ligne.

\subsubsection{Voitures}

Dans les voitures, il y a un bus principal CAN avec des sous-bus de type LIN.
Les constructeurs ne veulent généralement pas mettre à jour les micrologiciels de manière accessible, privilégiant une protection maximale, par exemple en utilisant de la mémoire non volatile qui ne peut être écrite qu'une seule fois.
Cependant, certaines entreprises comme Tesla ont la capacité de mettre à jour le micrologiciel de leurs voitures.

\subsection{Taille du microcontrôleur}

Il serait souhaitable d'utiliser un microcontrôleur de petite taille, par exemple avec des dimensions de 3x3mm ou 4x4mm.
Sur le circuit imprimé, il est possible de placer des composants des deux côtés.
Il est également nécessaire de choisir une tension d'alimentation, soit 3.3 Volts, soit 5 Volts.
Il faudra également décider si un connecteur sera choisi dans le cadre du projet.
Il faut également se questionner sur la question de la terminaison de la ligne.

\section{Vendredi 17 mars 2023}

Dans le cadre de ce projet, une marge de 10\% pour les imprévus serait suffisante au lieu des 20\% initialement proposés.
La difficulté principale du projet résidera dans la réalisation du chargeur de démarrage. Une des approches envisagées pour le projet est de commencer par réaliser l'interface utilisateur souhaitée.
Cela permettra de déterminer les étapes à suivre et les résultats à atteindre.

\begin{listing}[!h]
    \begin{lstlisting}[language=python]
        >>> import royal as r
        >>> r.discover()
        Found 2 buses: 
          - I2C /dev/bus/smbus-1
          - CAN /dev/...
        Found 3 devices:
          - /i2c/0/123 as color_sensor_left
            - Bootloader version 1.23
            - Architecture ARM Cortex M0+ 
            - Firmware abe17438738321
            - Firmware source http://github.com/
        >>> c = r.connect('color_sensor_left')
        ColorSensor instance
        >>> c.check_version('1.^')
        True
        >>> c.version
        1.23
        >>> c.color
        blue
        >>> c.turn_on_led()
        >>> c.
    \end{lstlisting}
    \caption{Exemple de résultat final}
\end{listing}

\section{Vendredi 31 mars 2023}

En termes de compatibilité, il serait souhaitable d'être compatible avec un maximum de microcontrôleurs.
Pour ce faire, un graphique en camembert pourrait être réalisé pour présenter les microcontrôleurs les plus utilisés, en privilégiant ceux basés sur l'architecture ARM.
De plus, un autre graphique en camembert pourrait être créé pour représenter les différents constructeurs de microcontrôleurs.
Enfin, un dernier graphique en camembert pourrait être élaboré pour illustrer l'utilisation des langages de programmation les plus courants sur les microcontrôleurs, tels que le C, le C++ et Rust.

Il convient également de prendre en considération les microcontrôleurs tels que AT SAM(T), ATmega et MSP430 pour évaluer leur compatibilité avec le projet.
En ce qui concerne les boitiers, il a été noté que le SO(ic)8 est trop volumineux, le (T)SSOP n'est pas intéressant, le QFN est intéressant, le TQFP est également intéressant, tandis que le BGA n'est pas soudable à la \gls{heig}.

Concernant le langage de programmation, il est courant que les microcontrôleurs soient codés en C.
Il serait également utile de vérifier la possibilité d'utiliser du C++ et d'explorer la piste du Rust, tout en gardant en tête qu'il ne faut pas s'engager directement dans cette voie.

\section{Vendredi 12 mai 2023}

Il est nécessaire de remplir la cause de non-confidentialité et de l'envoyer comme prévu.

Le journal de travail n'est pas obligatoire, mais il est toujours recommandé de le tenir avec rigueur. Faire un journal de travail incomplet n'apportera pas les bénéfices escomptés.

Il est important de vérifier la date de rendu du travail de bachelor et d'envoyer la date de fin pour validation.

Dans le rapport, il faudra inclure un bilan sur les avantages et les désavantages des frameworks utilisés.

Concernant les connecteurs, il serait pertinent de se renseigner sur les MOLEX CLIK-Mate. Ce sont de bons connecteurs, mais il convient de vérifier s'ils ne sont pas trop gros pour le cadre de ce projet.

Envisager la mise en place d'un chien de garde (\textit{watchdog}) pourrait être une option intéressante.

\section{Vendredi 26 mai 2023}

Pour déboguer le circuit imprimé qui sera réalisé, il est recommandé de se renseigner sur le Tag Connect et le SOICbite.

Il est important de noter que la commande Linux i2cdetect peut réagir différemment pour différentes adresses I2C en raison de l'alternance entre l'écriture et la lecture selon les adresses I2C.

Concernant le déploiement du nouveau micrologiciel, une approche intéressante pourrait être d'envoyer l'adresse de destination, la taille du bloc et les données.

Voici un aperçu de ce qui pourrait être fait lorsque l'utilisateur cherche à lister les périphériques I2C compatibles avec l'écosystème.

\begin{listing}[!h]
    \begin{lstlisting}[language=bash]
        $ ./jordil-master lookup
        #  uuid               i2c  firm   group ontology        name
        1. 0x123456789abcdef0 0x22 v1.1.2 1     sensor/light    Capteur couleur gauche
        2. 0x2a2a2a2a2a2a2a2a 0x43 v2.3.1 2     actuator/servo  Servo bras droit
        $ ./i2cdetect
        $ ./jordil-master upgrade --group2 http://git... 
        ----------> 100%
        OK
        
        
        git clone robot
        cd robot
        make all
        make deploy
    \end{lstlisting}
    \caption{Exemple de recherche de périphériques I2C compatibles avec l'écosystème}
\end{listing}

\section{Vendredi 2 juin 2023}

La taille du circuit imprimé devrait être d'environ 2.7 cm. Pour la commande, un circuit imprimé de catégorie 6C avec de préférence deux couches, mais quatre couches sont également acceptables.

Une autre carte Nucleo est disponible avec un microcontrôleur plus proche de celui choisi, mais elle ne semble pas nécessairement plus appropriée que celle actuellement utilisée, à moins qu'une raison valable ne soit avancée.

La date de rendu intermédiaire est prévue après environ 150 heures de travail, soit le 21 juin 2023.

\section{Mercredi 21 juin 2023}

Le rendu du rapport intermédiaire doit être effectué dans l'après-midi.
Le circuit imprimé doit également être passé en revue par Monsieur Yves Chevallier.