Dans cette annexe, il est exposé ce qu'est une licence de code et à quelles fins elle est utilisée.
De plus, il est précisé quelle licence a été sélectionnée et les raisons qui ont motivé ce choix.

\section{Licence}

Le choix de la licence informatique pour ce projet a été crucial pour déterminer les droits d'utilisation et de partage du code source.
Plusieurs licences open source étaient envisageables, chacune ayant ses particularités.
Parmi les licences les plus courantes, les licences GPL (General Public License), Apache, BSD et MIT ont été examinées.

La licence GPL est réputée pour son aspect copyleft, qui requiert que tout logiciel dérivé du code source sous licence GPL soit également publié sous la même licence.
Cette caractéristique garantit la libre distribution des logiciels dérivés, mais peut entraîner des restrictions pour les développeurs qui souhaitent intégrer le code dans des projets propriétaires.

La licence Apache, quant à elle, offre une plus grande flexibilité en permettant aux logiciels dérivés d'être distribués sous une licence différente, y compris propriétaire.
Cependant, elle impose certaines obligations, notamment de fournir une copie de la licence Apache avec tout logiciel dérivé.

La licence BSD est également permissive, mais comporte une clause publicitaire qui requiert l'attribution des droits d'auteur d'origine dans le code source et la documentation.
Cette licence est moins contraignante que la GPL, mais peut être perçue comme plus complexe.

En fin de compte, la licence MIT a été choisie pour sa simplicité et sa permissivité.
Elle permet une utilisation, une modification et une redistribution du code source sans imposer de contraintes importantes.
Sa compatibilité avec d'autres licences open source en fait un choix attrayant pour favoriser la collaboration et le partage dans la communauté open source, tout en donnant aux utilisateurs la liberté d'intégrer le code dans des projets propriétaires s'ils le souhaitent.
Cela permet d'encourager la réutilisation du code et de faciliter sa diffusion dans le plus grand nombre de projets possible.